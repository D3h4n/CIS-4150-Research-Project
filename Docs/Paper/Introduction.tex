\section{Introduction}
\label{Introduction}

In the modern era, software has become ubiquitous in human life. It can be found in cell phones, computers, vehicles, etc. Due to technological advancements such as the internet, it has become more important for companies to rapidly iterate on and update their software products to keep up with changing requirements from customers, stay ahead of competitors and maintain security. The concept of continuous integration and continuous delivery (CI/CD) was created to aid in solving these issues by emphasizing rapid delivery of changes to customers via automation. Automated software testing is one important piece of CI/CD which is used to maintain high quality software and reduce the risk of software failures. However, typically as software increases in scale and complexity, running automated test suites can take several hours. This often hinders software engineers in their development lifecycle especially when errors are discovered. Each fix can take hours to validate before additional changes can be made, resulting in significant delays when releasing updates.

Typically, for CI/CD, companies use systems such as Jenkins, GitHub Actions or GitLab CI. While these technologies support the connection of multiple servers to handle several workloads simultaneously, often, automated test suites are executed on a single computer per pipeline. One potential solution is to increase the number of tests that can be run at the same time and the speed at which these tests are executed using Distributed Computing. Distributed Computing in this context refers to spreading workloads such as automated tests across several computers, potentially in different physical locations, to increase the amount of work that can be done in parallel. To that end, this paper seeks to answer to what extent can test execution leverage distributed computing to improve efficiency and speed, considering factors such as memory and CPU usage, execution time, the trade-offs between vertical and horizontal scaling, and cost-effectiveness at scale.

The remaining sections of this paper are as follows: The next section will provide some background information on distributed computing and using this strategy for testing. In Section~\ref{Methodology}, we describe the design of our solution and the methods we used to measure and compare our solution to traditional approaches. In the Section~\ref{Findings}, we will present the results of our quantitative analysis which will then be further analyzed in Section~\ref{Discussion}. Finally, Section~\ref{Conclusion} will conclude the paper and propose potential future paths that can be explored.