\section{Discussion}
\label{Discussion}

For the smaller library (JSoup), it was found that increasing the number of containers has no impact on test execution. All but one configuration has no statistically significant difference. In the case that a difference was observed (2G Memory, 4 CPUs), the execution timings increased. This is likely due to the actual test execution not being a significant portion of the execution time. As a result, the overhead of creating the Docker containers resulted in an increase in time. Also, no significant improvement in execution time was found when increasing memory resources. It is likely that the lowest amount of memory provided was sufficient and hence, adding more memory had no effect. Finally, increasing the CPU resources overall improved test execution times. The transition from 1 CPU to 2 CPUs observed the largest increase, likely reflecting a lack of resources which may significantly slow down the test execution. The transition from 2 CPUs to 4 CPUs only saw an improvement for a single container whereas for 2 and 3 containers there was no further improvement. Much like memory scaling, this is likely due to the test execution times contributing to smaller percentages of the overall execution time and the resulting time being largely a result of the overhead of running each container. Furthermore, the data seems to suggest that at 3 containers, transitioning from 2 CPUs to 4 CPUs results in higher test execution times. This is due to reaching the resource limits of the testing computer which may have impacted the performance of the CPU. More investigation is needed to fully confirm or deny this. 

For the larger library (Guava), some evidence was found suggesting that increasing the number of containers improved test execution for specific configurations. In most instances, however, there is a lack of significant evidence to draw this conclusion. Regardless, the data from this paper’s experimentation shows a pattern of improvement based on the plots. With more testing, the significance of the differences between containers could possibly be improved. Much like the JSoup library, no evidence was found suggesting that increasing the amount of memory impacts test execution times. This draws a similar conclusion that sufficient memory has already been provided for test execution and further increases hold no impact. Furthermore, increasing the number of CPUs improved test execution times. Running with 1 and 2 containers produces significant evidence indicating each increment in CPU results in lower test execution times. In contrast, running with 3 containers yields no significant improvement. This suggests there may be an interaction between the number of containers and CPUs. It is also likely that adding more resources results in diminishing returns on improvement and increasing the number of containers reduces the impact of increasing the number of CPUs per container. 
